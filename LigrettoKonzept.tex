%\documentclass{templatearticle}
% -----------------------------------------------------------------------------
% User settings
% -----------------------------------------------------------------------------
%\myDoctype {}
%\myDivision {}
%\myTitle {}
%\mySubTitle {}
%\myAuthor {Rolf Koch}
%\myVersion {1.0}
%\myLanguage{german} % german or english
% -----------------------------------------------------------------------------
% Document: provide the two files abstract.tex and history.tex
% -----------------------------------------------------------------------------
%\begin{document}
%\setupLanguage            % setup configured language
%\setupWatermarks          % setup watermarks
%\printTitlePage{abstract} % include "abstract.tex"
%\printLeader{history}     % include "history.tex"
% -----------------------------------------------------------------------------
% Input user files with LaTex text sections
% -----------------------------------------------------------------------------

\documentclass[a4paper,oneside,notitlepage]{scrartcl}
\pdfcompresslevel=9



% define input encoding
\usepackage{ucs}
\usepackage[utf8]{inputenc}



%German Language
\usepackage[ngerman]{babel}

\newcommand{\authors}{Rolf Koch, Sandro Ropelato, Michael Schwarz, \\Andreas Ruckstuhl, Christof Würmli}
\newcommand{\batitle}{Ligretto}


% for hyperlinks (TODO: configure hyperref package)
\usepackage{hyperref}
\hypersetup{
	colorlinks=true, 
	linkcolor=blue,
	citecolor=blue,
	filecolor=blue,
	menucolor=black,
	%   pagecolor=black,
	urlcolor=blue,
	breaklinks=true,
	pdfnewwindow=true,
	pdfauthor=\authors,
	pdftitle=VSY: \batitle
	%   pdfcreationdate=D:\pdfdate,
	%   pdfmoddate=D:\pdfdate,
	%   pdfsubject=pdfsuject,
	%   pdfkeywords=pdfkeywords
}

% to include a PDF directly into the generated document
\usepackage{pdfpages}

% Marking entries as Todo with \todo{xyz}
\usepackage{todo}

% Für Syntax Highlightings
\usepackage{listings}

\lstset{ %
language=Octave,                % the language of the code
basicstyle=\footnotesize,       % the size of the fonts that are used for the code
numbers=left,                   % where to put the line-numbers
numberstyle=\footnotesize,      % the size of the fonts that are used for the line-numbers
stepnumber=1,                   % the step between two line-numbers. If it's 1, each line 
                                % will be numbered
numbersep=4pt,                  % how far the line-numbers are from the code
backgroundcolor=\color{white},  % choose the background color. You must add \usepackage{color}
showspaces=false,               % show spaces adding particular underscores
showstringspaces=false,         % underline spaces within strings
showtabs=false,                 % show tabs within strings adding particular underscores
frame=single,                   % adds a frame around the code
tabsize=2,                      % sets default tabsize to 2 spaces
captionpos=b,                   % sets the caption-position to bottom
breaklines=true,                % sets automatic line breaking
breakatwhitespace=false,        % sets if automatic breaks should only happen at whitespace
title=\lstname,                 % show the filename of files included with \lstinputlisting;
                                % also try caption instead of title
escapeinside={\%*}{*)},         % if you want to add a comment within your code
morekeywords={*,...}            % if you want to add more keywords to the set
}

% for \colorbox
\usepackage{color}
% for graphics
\usepackage{graphicx}


% Big Tables across Pages
\usepackage{longtable}

\begin{document}
% titelseite
\begin{titlepage}
% \begin{center}
%  \includegraphics[width=0.3\textwidth]{graphics/zhaw_title_logo.png} \\[1cm]
%
%  Bachelorarbeit \\ {\large \batitle} \\[1cm]
%
%  \includegraphics[width=0.4\textwidth]{graphics/infrasupport_logo.jpg}
%  \includegraphics[width=0.1\textwidth]{graphics/suw_logo.jpg}
% \end{center}

% standard titelseite:
\title{VSY Konzept \\ \batitle}
\author{\authors}
\date{24. April 2011}
\maketitle
\end{titlepage}

\pagenumbering{roman} 
%\input{0_zusammenfassung} \newpage
% Inhaltsverzeichnis
\renewcommand*\contentsname{Inhaltsverzeichnis}
\tableofcontents \newpage

\pagenumbering{arabic}
% content:
\section{Aufgabenstellung} 

Der Verlauf eines Ligretto-Spiels soll simuliert werden können. Um echten Zufall ins Spiel zu bringen, soll das Spiel über mehrere Maschinen verteilt werden.

 \begin{itemize}
 \item Auf dem \textbf{Eingangsserver} lassen Sie einen Webserver laufen. Ein POST-Request auf den Server initiiert eine Ligretto-Simulation und liefert einen Link über den das Resultat per Polling abgholt werden kann. Das Resultat der Simulation wird  unter diesem Link als XHMTL-Seite präsentiert. Während der Berechnung wird eine Seite mit dem Text \textit{in Bearbeitung} angezeigt.
 \item Eingangsparameter: Ausgangslage des Spiels (Verteilung der Karten) und implizit die Anzahl Spieler. Rückgabewert: Punkteverteilung für die Spieler
 \item Dieser Eingangsserver sei bereits komplett definiert. Sie erhalten die Daten aus dem POST-Request in der von Ihnen gewünschten Datenstruktur zur Weiterverarbeitung.
 \item Um das \textbf{Mischeln und Verteilen} der Karten müssen Sie sich nicht kümmern. Der Eingangsserver liefert ja bereits die Spielsituation genau vor Beginn des Spiels. Zu diesem Zeitpunkt sind die Karten verteilt und Spezialregeln (neue Karten bei $>= 30$ Punkte) sind bereits erfüllt. Aber Sie müssen dafür Sorgen, dass die \textbf{Clients die nötigen Informationen vom Eingangsserver} erhalten. Dies ist Abhängig von der von Ihnen gewählen Implementation.
 \item Ihnen stehen für die Simulation eine bestimmte \textbf{Anzahl Rechner} zur Verfügung. Die Rechner haben sich vorgängig beim Eingangsserver gemeldet, so dass er eine Liste der verwendbaren Clients hat und damit die Anzahl der verwendbaren Rechner kennt. Sie brauchen sich nicht darum zukümmern wie sich die Clients beim Server anmelden.
 \item Die \textbf{CPU-Leistung} ist nicht zu optimieren.
 \item Die \textbf{Berechnungszeit} ist  nicht zu optimieren.
 \item Beschreiben Sie den \textbf{Algorithmus} vollständig (Ax im Bewertungsmassstab). Sie beginnen ab dem Moment wo Sie die Daten vom Eingangsserver erhalten (in der von Ihnen definierten Datenstruktur). Das erste wird wohl die Verteilung der Daten auf die Client-Rechner seind.
 \item Beschreiben Sie alle verwendeten \textbf{Datenstrukturen} inklusive der Deklaration in Java oder C/C++.
 \item Beschreiben Sie die \textbf{Schnittstellen} als RPC x-File oder RMI-Interface. Nur die Schnittstellen die während dem Spiel benutzt werden sind hier verlangt.
 \item Die Anzahl Ziffern (1-10) und Frontfarben (4 verschiedene) sind unveränderliche Konstanten
 \item Wie \textbf{skaliert} das System mit Zunahme der Anzahl Spieler?
 \item Die restlichen Bewertungskriterien (wie Rechnerausfall etc.) können Sie weglassen
 \item Die \textbf{Spielstrategie} können Sie an eine Funktion KI auslagern. Wenn Sie also aus spieltechnischen Ueberlegungen eine Karte nicht legen wollen, obwohl es möglich wäre, so kann diese Entscheidung durch diese KI-Funktione ermittelt werden. Sie müssen allerdings die für die Entscheidung nötigen Informationen dieser Funktion zur Verfügung stellen. D.h. Sie müssen sich überlegen welche Informationen nötig sind und woher diese Informationen besorgt werden. Aber um die Implementation der Spieletheorie müssen Sie sich nicht kümmern.
\end{itemize}
	
	
% Beispiel um eine Grafik einzubinden
	
%\begin{figure}[hbt]
%  \centering
%  \includegraphics[width=0.65\textwidth,angle=0]{graphics/zhaw.png}
%  \caption{ZHAW Logo}
%  \hfill{} }
% \end{figure}
 \newpage
\section{Spielbeschreibung}

\begin{figure}[hbt]
  \centering
  \includegraphics[width=0.45\textwidth,angle=0]{graphics/ligretto.jpg}
  \caption{Ligretto \hfill{} }
 \end{figure}

Ligretto ist ein Spiel bei dem es darum geht seinen eigenen Ligretto Stapel los zu werden bevor dies ein Gegner schafft. Der Ligretto Stapel besteht aus 10 verdeckten zufälligen Karten aus dem Kartenvorrat des Spielers. Der Kartenvorrat eines Spielers, auch Deck genannt, besteht aus 40 Karten. Die Karten sind von 1 bis 10 nummeriert und haben 4 Farben: Rot, Gelb, Blau und Grün. Es gibt keine doppelten Karten, das heisst jede Karte kommt nur genau einmal vor im Deck des Spielers.

\begin{figure}[hbt]
  \centering
  \includegraphics[width=0.20\textwidth,angle=0]{graphics/ligretto.png}
  \caption{Ligretto Karten \hfill{} }
 \end{figure}

Um ein neues Spiel vorzubereiten mischt jeder Spieler seinen Kartenstapel. Jeder Spieler zählt dann 10 Karten ab und legt diese verdeckt vor sich hin. Dies ist sein eigener Ligretto Stapel. Danach legt er 4 Karten von seinen restlichen Handkarten offen neben den Ligretto Stapel. Sobald alle Spieler soweit sind geht es los.

Der Spielablauf funktioniert nun wie folgt für jeden Spieler:
\begin{enumerate}
\item Der Spieler schaut die offen vor sich liegenden Karten an.
\item Dann prüft der Spieler die Stapel an Karten in der mitte des Spielfelds. 
\item Der Spieler versucht nun eine Karte zu finden, die eins höher ist als die oberste Karte eines Stapels und die selbe Farbe besitzt. Ist dies der Fall, so darf er seine Karte nehmen und oben auf diesen Stapel legen.
\item Wenn diese gelegte Karte eine der 4 offen vor sich liegenden Karten gewesen ist, dann darf der Spieler die oberste Karte vom Ligretto Stapel aufdecken und anstelle der soeben gelegten Karte vor sich hin legen.
\item Wenn eine 1er Karte verfügbar ist, darf damit ein neuer Stapel in der Mite des Spielfelds eröffnet werden.
\item Wenn mit den offenen Karten keine Aktionen durchgeführt werden können, dann darf der Spieler von seinen restlichen Handkarten 3 Karten verdeckt abzählen und diese umdrehen und vor sich auf den Ablagestapel legen. Existiert noch kein Ablagestapel so legt er die 3 Karten einfach neben seine bereits vor sich liegenden Karten und eröffnet somit seinen Ablagestapel. Beim Ablagestapel darf nun jeweils die oberste Karte ebenfalls gespielt werden.
\item Sollte weiterhin keine Karte gelegt werden können, so wiederholt der Spieler Punkt 6 und legt immer 3 Karten oben auf den Ablagestapel, bis er keine Handkarten mehr in der Hand hält. Ist dies der Fall, so nimmt der Spieler den Ablagestapel wieder als Handkarten auf und fängt von vorne an.
\end{enumerate}

Sobald ein Spieler die letzte Karte seines Ligretto Stapels aufgedeckt hat, ruft er Ligretto Stop und das Spiel wird beendet und es wird abgerechnet.

Bei der Abrechnung bekommt jeder Spieler Punkte für sich selber. Die Abrechnung läuft wiefolgt ab:


\begin{enumerate}
\item Jeder Spieler zählt seine noch vorhandenen Ligretto Stapel Karten. Für jede Karte auf dem Ligretto Stapel bekommt er zwei Minuspunkte.
\item Die Karten auf dem Spielfeld werden nun wieder nach Farbe auf der Rückseite sortiert. Jeder Spieler erhält seine Karten zurück und zählt diese. Für jede gelegte Karte bekommt der Spieler einen Pluspunkt.
\item Die Punkte werden nun notiert und dann fängt die nächste Runde an.
\end{enumerate}  \newpage
\section{Problemanalyse}   %wo liegen die Probleme oder Schwierigkeiten in einer Ligrettosimulation mit verteilten Systemen?

%allgemein, was muss analysiert werden

Als Aufgabe gilt es, den Ablauf, der durch die Spielregeln festgelegt ist, anhand folgender Gesichtspunkte in einen Algorithmus umzusetzen:


\begin{enumerate}
	\item Der Spielablauf muss den Spielregeln entsprechen und die KI-Algorithme, welche die Mitspieler simulieren, sollten nur an Informationen über den Spielablauf gelangen, welche auch ein menschlicher Spieler gelangen würde.
	\item Der Spielablauf sollte möglichst fair sein. D.h. bis auf den Zufall, der durch das Mischen der Karten ins Spiel gelangt, sollten alle Mitspieler gleichgestellt sein.
	\item Die verwendeten Algorithmen und Schnittstellen sollten möglichst gut mit der Anzahl Mitspieler skalieren, unter Betrachtung der benötignen Rechenzeit und Netzerkkapazität.
\end{enumerate}

Folgende Punkte werden bei der Herleitung des Algorithmus nicht betrachtet:

\begin{enumerate}
	\item Alle Mitspieler müssen sich an die Spielregeln halten, insbesonder in Teilen des Ablaufs, die im Spiel mit menschlichen Spielern nicht überwacht werden können. Dazu gehör z.B. das regelkonforme Ablegen der Handkarten beim suchen einer passenden Karte.
	\item Die beiligten Server versenden nur Nachrichten an Server, mit denen sie laut dem Alogrithmus kommunizieren dürfen.
	\item Die eingesetzte Software und Hardware darf keine Fehler aufweisen und der Netzerkverkehr darf nicht gestört oder unterbrochen werden. D.h. z.B. keine Pakete dürfen verloren gehen.
\end{enumerate}

%Für die letzten beiden Punkte wird anschliessend eine Diskusion gegeben, inwiefern der Alogorithmus und die Schnittstellen angepasst werden müssten, um ...

\subsection{Systemgrenzen}

%welche teile des algorithmus laufen wo

Damit die Laufzeit eines Spiels nicht, oder nur schwach, mit der Anzahl Spieler wächst, muss die Anzahl verwendeter Server linear mit der Anzahl Mitspieler wachsen können. Deshalb ist es notwendig und sinnvoll für jeden Mitspieler mindestens einen, separaten Prozess zu Verwenden.

Da ein einzelner Spieler jedoch zu jeder Zeit immer nur eine Aktion ausführen darf (siehe Spielregeln), ist eine Aufteilung eines einzelenen Spielers in mehrere Prozesse nicht sinnvoll.

Zur Überwachung des gesammten Spiels, sowie dessen Aufbau und Abbau, wird ein separater Prozess verwendet. Da der Überwachende Prozess nur vor Spielbeginn und nach Beendigung des Spiels eine Aufgabe hat, dessen Komplexität mit der Anzahl mitspieler wächst, ist eine Aufteilung deiser Aufgaben in mehrere Prozesse zwar erwägenswert, aber nicht von oberster Priorität.

% TODO: Nomenklatur für den Überwachende Prozess und Mitspieler-Prozesse

Zusammengefasst:

\begin{enumerate}
	\item Ein Prozess zur Überwachung des Spiels (Aufbau und Abbau).
	\item Ein Prozess pro Mitspieler, welche alle Aktionen des jeweiligen Mitspielers durchführt.
\end{enumerate}

\subsection{Verteilung}

%welcher state ist wo

Bei der Betrachtung der Verteilung des Zustandes gibt es drei Arten von Objekten:

\begin{enumerate}
	\item {\bf Zustands-Objekte} verkörpern den veränderlichen Zustand eines Spiels. Die Eigenschaften dieser Objekte sind somit mutierbar. Für diese Objekte ist immer genau ein Prozess zuständig und diese Objekte werden auf andere Prozesse verschoben oder kopiert.
	\item {\bf Token-Objekte} werden zur Kommunikation verwendet und sind nicht weränderbar. Token-Objekte können zwischen den Prozessen verschoben werden, jedoch ist immer nur genau ein Prozess für ein jeweiliges Token-Objekt verantwortlich.
	\item {\bf Transiente Objekte} werden zu Kommunikations-Zwecken und zur unterstützung der Algorithmen erstellt und aufbewahrt. Wenn ein transientes Objekt an einen anderen Prozess versendet wird, dann besitzen beide Prozess eine eigene Kopie des Objektes.
\end{enumerate}

Die folgenden Token-Objekte können identifiziert werden:
\begin{enumerate}
	\item Spielkaren: Zu beginn des Spieles wird für jede Spielkarte, welche in einen realen Ligretto-Spiel verwerdet würde, ein Spielkarten-Objekt erstellt.
\end{enumerate}

Folgende Zustands-Obekte können identifiziert werden:
\begin{enumerate}
	\item Handkarten-Stapel
	\item Ausgelegte Karten vor dem Mitspieler
	\item Stapel auf dem Spieltisch
	\item Mitspieler
	\item Überwachender Prozess
\end{enumerate}

Zur Kommunikation werden folgende Transiente Objekte erstellt:
\begin{enumerate}
	\item Start- und Stop-Signale des überwachenden Prozesses.
	\item Stop-Signal eines Mitspielers.
	\item Stapel mit gemischten Karten, welche vom Überwachenden Prozess and die Mitspieler verteilt werden.
	\item ...
\end{enumerate}

\subsection{Kommunikation}

Bei der definition der Kommunikation muss gleichzeitig die Korrektheit und die Performance beachtet werden.

Der Ablauf lässt sich in folgende Abschnitte unterteilen:
\begin{enumerate}
	\item Während dem {\bf Aufbau} werden die Spielkarten durch den Server vorbereitet und and die regisrierten Clients verteilt.
	\item Stop-Signal eines Mitspielers.
	\item Stapel mit gemischten Karten, welche vom Überwachenden Prozess and die Mitspieler verteilt werden.
	\item ...
\end{enumerate}


%für was wird wie kommuniziert

\subsubsection{Vor Spielbeginn}

Ausgangslage bei Spielbeginn ist folgende:
\begin{itemize}
\item Der Server kennt alle Spieler, die an dieser Spielrunde teilnehmen.
\item Dem Server stehen eine entsprechende Anzahl gut gemischter Kartendecks zur Verfügung.
\item Kommunikationsgrundlagen (Netzwerk, RMIRegistry, etc...) sind gegeben.
\end{itemize}

\subsubsection{Spielinitialisierung}
Zuerst initialisiert der Server alle ihm bekannten Spieler, indem er über RMI deren init-Methode aufruft. Dabei werden sämtliche Informationen, die ein Spieler kenne muss, übermittelt. (Referenz zum Server und zu den Mitspielern, die gemischten Karten)

\subsubsection{Spielstart}
Nach der Initialisierung werden alle Spieler über den Spielstart informiert. Dies geschieht über den Remoteaufruf der Methode spielStart.

\subsubsection{Während des Spiels}
Sobald ein Spieler über den Spielstart informiert wurde, versucht er der Reihe nach die Karten aus seinen Spielslots oder jene aus seinem Handstapel auf einen eigenen oder den Stapel eines Nachbarn zu legen. Für einen Versuch auf dem eigenen Stapel ist kein Remoteaufruf nötig. Um die Karte bei einem Nachbarn zu platzieren, wird diesem die Karte mittels Aufruf der Methode legeKarte übergeben. Hat die Karte auf einem Stapel platzgefunden, so bestätigt das der Empfänger der Karte mit true, ansonsten gibt er false zurück und die Karte geht wieder zurück in den Slot bzw. in den Handkartenstapel.

\subsubsection{Ligretto Stop!}
Ist der Ligrettostapel eines Spielers leer, so ruft er die ligrettoStop-Methode des Servers auf. Dieser übermittelt diese Botschaft allen Spielern, indem er deren spielEnde-Methode aufruft.

\subsubsection{Punkte zählen}
Nach Spielende, ruft der Server von allen Spielern den Punktestand ab. Dies geschieht über dem RMI-Aufruf der Methode getPunkte.


\subsection{Concurrency-Problem}

\subsubsection{Gleichzeitiges Legen einer Karte}
Es kommt vor, dass zwei Spieler gleichzeitig versuchen die selbe Karte auf den selben Stapel zu legen. Wie im echten Spiel gilt auch in der Spielsimlulation: "First come first served". Dabei werden alle RMI-Methoden mit dem \textbf{synchronized}-Schlüsselwort synchronisiert.

\subsubsection{Spielstop währenddem eine Karte gelegt wird}
Wird das Spiel während einem Versuch, eine Karte auf den Stapel zu legen beendet, darf die Karte gemäss Spielregeln nicht mehr gelegt werden. Dies kann erreicht werden, indem man die Erfolgsmeldung mit dem Spielstatus verknüpft. Ist das Spiel beim Aufruf der legeKarte-Methode beendet, wird in jedem Fall false zurückgegeben.

\subsection{Zuverlässigkeit}

\subsubsection{Verlorenes Netzwerkpaket}
Da wir uns für den Einsatz von RMI entschieden haben, brauchen wir uns um Paketverlust etc. keine Gedanken zu machen. RMI benutzt TCP zur Kommunikation und stellt somit die Übermittlung einer Nachricht sicher.

\subsubsection{Absturz eines Spielers}
Stürzt ein Spieler während des Spiels ab, so beeinträchtigt das den Verlauf des Spiels nicht. Um eine lange Antwortzeit beim Legen einer Karte zu vermeiden werden ihn seine Nachbarn für weitere Anfrageversuche ignorieren.

\subsubsection{Absturz des Servers}
Da der Server für die Übermittlung der Ligretto-Stop-Nachricht verantwortlich ist, hätte ein Serverabsturz zur Folge dass alle Spieler Spielen, bis sie ihren Ligrettostapel abgearbeitet haben, und dann beim Aufruf der ligrettoStop-Methode erfahren, dass der Server keine Antwort mehr gibt. Der Spielausgang wäre damit verfälscht und nicht brauchbar.

\subsection{Skallierbarkeit}

Gemäss Spielregeln braucht es mindestens zwei Spieler für ein Ligrettospiel. Gegen oben existieren keine Grenzen. Dadurch dass nur eine Kommunikation mit einer konstanten Anzahl Nachbarn erfolgt, bleibt die Netzwerkauslastung und der Rechenaufwand pro Spieler auch bei steigender Anzahl Gegner konstant. Je nach Implementation können Nachrichten, welche an alle Spieler gehen (spielStart und spielEnde) durch Grenzen der Netzwerkgeschwindigkeit etwas verspätet eintreffen.

%Bewertungsmassstab:
% Bei Ligretto berücksichtigen, dass die Anzahl Spieler steigen kann.
%    - Ist die Zahl der beteiligten Rechner definiert? Wieviele Rechner sind minimal nötig, wieviele können max. eingesetzt werden?
%    - Wie steigt die Leistung des Systems mit hinzugefügten Rechnern? (Bis zu 1 Zusatzpunkt für detaillierte Analyse)
 \newpage
\section{Lösungsansatz} 

Wir benutzen RMI. \newpage
\section{Spiel Algorithmus} 

\subsection{Strategie}

\subsection{KI}

Ist nicht direkt Teil unseres Konzepts, aber wir müssen der KI gewisse Informationen zur Verfügung stellen damit sie überhaupt operieren kann.

Dazu gehören:

Spiel Start Signal
Stapel in der mitte des Spielfelds
Die Handkarten des Spielers
Das Legen einer Karte auf einen Stapel (Erfolgreich oder nicht)
Spiel End Signal
Punkte Abrechnungs Informationen

\subsection{Ablaufdiagramm}

\begin{enumerate}
	\item dä Server verwaltet diä KartenStapel wo i dä mitti vom spielfeld sind
	\item dä Server hät am afang pro spieler 40 karten wo vorsortiert sind diä muäs är irgendwiä äm jewiligä spieler übergä
	\item sobald diä karte übergä sind übernimmt dä client d kontrolle über diä chartene
	\item dä client baut sin ligretto stapel, + sini 4 offne charte und seit dänn am Server, dass er Ready isch
	\item dä server wartet bis alli Ready sind und macht dänn äs Startsignal
	\item jede client fangt jetzt a regelmässig dä zuästand vo dä Stapel i dä Mitti abzfragä und versuächt sini chartene z spiele.
	\item Wenn er ä möglichkeit gseet versuächt er ä charte am Server z übergä für än gwüssä Stapel. Er wartet bis dä Server meldet ob das klappt hät oder nöd
	\item Sobald das klappt hät macht er wiiter, wänns nöd klappt hät chunt er vom Server sini Charte zrugg über und muäs si wieder detä härä legä wo si gsi isch.
	\item nachdem ein spieler sin ligretto stapel fertig hät macht er irgend äs Ligretto Stop signal. dä Server leitet das wiiter a alli spieler und returnt no alli charte wo erst nach däm ligretto stop signal acho sind.
	\item jede spieler zellt jetzt sin ligretto stapel und meldet diä zahl an Server
	\item dä Server sortiert alli karte, zellt si und verrechnet jetzt diä pünkt.
\end{enumerate}

\begin{figure}[hbt]
  \centering
  \includegraphics[width=0.80\textwidth,angle=0]{graphics/KarteLegen.png}
  \caption{Sequenzdiagramm zum Legen einer Karte [Papier und Stift FTW] \hfill{} }
 \end{figure}
 \newpage
\section{Datenstrukturen} 

\subsection{Klassendiagramm}

Karten

Handkarten

Kartenstapel

Collection aller Kartenstapel

Start Signal

End Signal

Abrechnungs Daten \newpage
\section{Schnittstellen} 

\subsection{RMI-Interfaces}

\subsubsection{Client}


\lstinputlisting[language=Java]{listings/RemoteClient.java}


\subsubsection{Server}

\lstinputlisting[language=Java]{listings/RemoteServer.java}


\subsubsection{Datenstrukturen}

\lstinputlisting[language=Java]{listings/Stapel.java}

\lstinputlisting[language=Java]{listings/Karte.java}


% Für die Direkte eingabe des Codes kann man das hier benutzen:
% \begin{lstlisting}[language=Java]
% \end{lstlisting}


 \newpage
\input{8_ErwartetesErgebnis} \newpage


% Appendices

%\appendix

%\section{Quellenverzeichnis}
% TODO: BiBTex??
%\bibliographystyle{unsrt}
%\bibliography{quellenverzeichnis}

\todos

\end{document}