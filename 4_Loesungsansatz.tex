\section{Lösungsansatz} 

Client 1 hat vom Server Client 2 bis Client 7 als Nachbarn zugewiesen bekommen. Als erstes selektiert der Client 1 eine Karte und versucht diese bei sich selbst Legen. Gelingt dies nicht, versucht er die Karte bei einem seiner 6 Nachbarn zu legen, indem er ihnen die Karte übergibt und, falls sie gelegt werden konnte, eine Erfolgsmeldung (\textit{true}) oder ansonsten eine Misserfolgsmeldung (\textit{false}) zurückbekommt. Konnte er die selektierte Karte legen, wird diese gelöscht und eine neue Karte wird selektiert, mit welcher wieder von vorne begonnen wird. Das Beispiel von Client 1 ist representativ für alle Clients. 

\begin{figure}[hbt]
  \centering
  \includegraphics[width=0.90\textwidth,angle=0]{graphics/Kartenlegen_Sequenzdiagramm.png}
  \caption{Sequenzdiagramm: Kartenlegen}
\end{figure}

% Bewertungsmasstab:
% - Ist der Algorithums zur Lösung der Teilafugaben nachvollziehbar? (3 Punkte)

%    - Der Teil für die Initialisierung/Setup wird nicht hier bewertet
%    - 2 Punkte für die komplexeren Komponenten (z.B. die "Clients")
%    - 1 Punkt für die weniger komplexen Komponenten
%    - Das umfasst auch die Kommunikation (wann kommuniziert wer mit wem)

% Nutzung der Ressourcen / Performance
%  -  Ist die Kommunikation minimiert?
%  -  Ist der Bedarf an Memory / Storage sinnvoll?